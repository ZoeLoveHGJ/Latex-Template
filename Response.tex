%%%%%%%%%%%%%%%%%%%%%%%%%%%%%%%%%%%%%%%%%%%%%%%%%%%%%%%%%%%
% Rho Class - Response to Reviewers Template
% --------------------------------------------------------
% 使用说明:
% 1. 填写论文信息(\papertitle, \journalname 等)
% 2. 开头为致编辑/审稿人的总结信
% 3. 目录自动生成,方便审稿人定位
% 4. 使用 \reviewer{N} 分区,每个问题用:
%    - reviewercomment 环境(审稿意见)
%    - authorresponse 环境(作者回复)
%    - revisedtext 环境(修改后的正文段落)
% 5. 可用 \added{新增文字}, \deleted{删除文字} 标记修改
% --------------------------------------------------------
%%%%%%%%%%%%%%%%%%%%%%%%%%%%%%%%%%%%%%%%%%%%%%%%%%%%%%%%%%%

\documentclass{class/response}

%----------------------------------------------------------
% 基本宏包配置
%----------------------------------------------------------
\usepackage[english]{babel}

%----------------------------------------------------------
% 主题色配置 (可选: red, purple, blue)
%----------------------------------------------------------
\rhotheme{red}

%----------------------------------------------------------
% 论文信息(定义一次,全局复用)
%----------------------------------------------------------
\papertitle{LODS-MTI: A Link-Adaptive, Orthogonal, and De-slotted Protocol for Robust and Fast RFID Missing Tag Identification}
\journalname{IEEE Transactions on Mobile Computing}
\authorname{Xiaolin Jia et al.}
\manuscriptid{TMC-2025-XXXXX}

%----------------------------------------------------------
\begin{document}

\makeresponsetitle

%==========================================================
% 致编辑与审稿人的总结信
%==========================================================

\textbf{Dear Editors and Reviewers:}

We would like to express our sincere appreciation to the Editor-in-Chief, the Associate Editor, and the Reviewers for their time and effort in reviewing our manuscript. We are particularly grateful for the encouraging comments from \textbf{Reviewer \#2} and \textbf{Reviewer \#4}, who recognized the novelty and readiness of our work.

Simultaneously, we found the critical questions raised by \textbf{Reviewer \#1} and \textbf{Reviewer \#3} regarding physical layer assumptions and worst-case scenarios to be insightful. These comments have guided us to perform a fundamental upgrade to our simulation framework and validation methodology.

We have carefully addressed all comments point-by-point. Our major revisions are summarized as follows:

\begin{enumerate}
    \item Re-executed ALL experiments using an upgraded high-fidelity simulation framework.
    \item Implemented a refined dynamic energy model for listening cost analysis.
    \item Introduced a Channel Middleware layer to model realistic physical layer imperfections.
    \item Conducted a comparative ablation study to verify our ``Earliest-Position-First'' strategy.
    \item Made the complete source code publicly available for transparency and reproducibility.
    \item Added a rigorous discussion on worst-case complexity; all major revisions are \added{highlighted in red} for easy tracking.
    \item Updated the references strictly according to IEEE journal requirements.
\end{enumerate}

We believe these extensive revisions have significantly strengthened the technical depth, rigor, and practical relevance of the paper. We hope the revised manuscript is now suitable for publication.

\vskip0.5em
Best regards,

\textbf{Prof. Xiaolin Jia and Co-authors} \hfill \today

%==========================================================
% 目录(方便审稿人快速定位)
%==========================================================

\vskip2em
\makeresponsetoc


%==========================================================
% Reviewer #1
%==========================================================

\reviewer{1}

% --- Comment 1.1 ---
\begin{reviewercomment}
The paper assumes perfect physical layer conditions. How does the protocol perform under realistic channel impairments such as timing jitter, sensing errors, and tag dropout?
\end{reviewercomment}

\begin{authorresponse}
Thank you for this insightful comment. We have introduced a Channel Middleware layer in our simulation framework to model realistic physical layer imperfections. Specifically, we now account for:

\begin{itemize}
    \item Timing jitter ($\pm 5\%$ variation in slot boundaries)
    \item Bit-level sensing errors (BER = $10^{-3}$)
    \item Tag dropout probability ($p_{\text{drop}} = 0.01$)
\end{itemize}

The revised results show that LODS-MTI maintains its performance advantage with less than 3\% degradation. Please see Section 5.3:

\begin{revisedtext}
To validate the robustness of LODS-MTI under realistic conditions, we introduce a Channel Middleware layer that models three types of physical layer imperfections: (1) \added{timing jitter with $\pm 5\%$ slot boundary variation}, (2) \added{bit-level sensing errors at BER = $10^{-3}$}, and (3) \added{stochastic tag dropout with probability $p_{\text{drop}} = 0.01$}. Our experimental results demonstrate that the protocol maintains its performance advantage with less than 3\% degradation compared to ideal conditions.
\end{revisedtext}

\end{authorresponse}

% --- Comment 1.2 ---
\begin{reviewercomment}
The worst-case complexity analysis is missing. What happens when the number of missing tags is very large (e.g., 50\% or more)?
\end{reviewercomment}

\begin{authorresponse}
We have added Section 4.4 analyzing the worst-case complexity:

\begin{revisedtext}
\textbf{Worst-Case Analysis.} When the missing rate $\alpha$ exceeds a threshold $\alpha_{\text{th}} = 0.5$, the communication overhead of incremental identification surpasses that of full re-inventory. In this regime, LODS-MTI \added{automatically triggers a fallback to complete re-inventory mode, which achieves $O(n)$ complexity where $n$ is the total tag population.}
\end{revisedtext}

\end{authorresponse}


%==========================================================
% Reviewer #2
%==========================================================

\reviewer{2}

% --- Comment 2.1 ---
\begin{reviewercomment}
Some references are not formatted according to IEEE standards. Please update accordingly.
\end{reviewercomment}

\begin{authorresponse}
Thank you. We have reformatted all references to comply with IEEE transaction standards:

\begin{itemize}
    \item Replaced informal citations with proper IEEE format
    \item Added DOI links where available
    \item Ensured consistent author name formatting
\end{itemize}
\end{authorresponse}


%==========================================================
% Reviewer #3
%==========================================================

\reviewer{3}

% --- Comment 3.1 ---
\begin{reviewercomment}
The energy model does not account for the listening cost during idle slots. This may lead to underestimation of total energy consumption.
\end{reviewercomment}

\begin{authorresponse}
Excellent observation. We have refined the energy model to include listening costs. The updated analysis (Section 5.2):

\begin{revisedtext}
The total energy consumption $E_{\text{total}}$ is computed as:
\begin{equation}
    E_{\text{total}} = \sum_{i=1}^{R} \left( E_{\text{tx},i} + \added{E_{\text{listen},i} \cdot N_{\text{active},i}} + E_{\text{idle},i} \right)
\end{equation}
where \added{$E_{\text{listen},i}$ represents the per-tag listening energy during round $i$, and $N_{\text{active},i}$ is the number of active tags in that round.}
\end{revisedtext}

\end{authorresponse}


\end{document}
