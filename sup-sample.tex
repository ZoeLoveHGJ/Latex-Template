%%%%%%%%%%%%%%%%%%%%%%%%%%%%%%%%%%%%%%%%%%%%%%%%%%%%%%%%%%%
% --------------------------------------------------------
% Supplementary Material Template
% --------------------------------------------------------
%%%%%%%%%%%%%%%%%%%%%%%%%%%%%%%%%%%%%%%%%%%%%%%%%%%%%%%%%%%

% 加载补充材料文档类 (单栏,9pt)
\documentclass[9pt,a4paper,twoside,onecolumn]{sup}

\usepackage[english]{babel}

% --------------------------------------------------------
% 自定义设置
% --------------------------------------------------------

% 重置计数器,使章节、图表编号带有 "S" 前缀 (S.1, Fig S1)
\renewcommand{\thesection}{S.\arabic{section}}
\renewcommand{\thefigure}{S\arabic{figure}}
\renewcommand{\thetable}{S\arabic{table}}
\renewcommand{\theequation}{S\arabic{equation}}

% --------------------------------------------------------
% 文章标题
% --------------------------------------------------------

\title{Supplementary Material for "Paper Title..."}

% 作者信息 (可选,如果不希望显示太多,可精简)
\author[1]{First Author}
\author[1,2]{Second Author}
\author[1]{Third Author}

\affil[1]{School of Computer Science and Technology, University Name, City, Country}
\affil[2]{Key Laboratory of Research, City, Country}

\leadauthor{Supplementary Material} % 页眉显示

% --------------------------------------------------------
% 开关
% --------------------------------------------------------
\setbool{rho-abstract}{true}
\setbool{corres-info}{false} % 补充材料通常不显示底部详细通信信息

% --------------------------------------------------------
% 摘要 (必须放在 \maketitle 之前)
% --------------------------------------------------------
\begin{abstract}
This supplementary document provides comprehensive mathematical proofs, detailed simulation settings, and additional experimental results that support the main claims of the paper.
\end{abstract}

% --------------------------------------------------------
% 正文
% --------------------------------------------------------
\begin{document}

\maketitle
\thispagestyle{firststyle} % 使用专用首页样式

% 目录
\tableofcontents
\vspace{20pt}
\hrule
\vspace{20pt}

% --------------------------------------------------------
% 内容区
% --------------------------------------------------------

\section{Detailed Derivations}
This section presents the derivation of the optimal frame length. By minimizing the collision probability, we can derive the condition for maximum throughput.

\section{Additional Experiments}
\subsection{Throughput Analysis}
We further analyze the system throughput under different tag densities.

\subsubsection{Impact of Velocity}
The tag velocity introduces Doppler shifts, which may affect the decoding performance. However, our results show that our protocol remains robust up to 5 m/s.

% --------------------------------------------------------
% 宽表格示例 (单栏模式优势)
% --------------------------------------------------------
\begin{table}[h]
    \centering
    \caption{Detailed Simulation Parameters (Full Width)}
    \rowcolors{2}{gray!10}{white}
    \begin{tabular}{lcccc}
        \toprule
        \textbf{Parameter} & \textbf{Value} & \textbf{Unit} & \textbf{Description} \\
        \midrule
        Tag Population & 100 -- 1000 & - & Number of tags in SOA \\
        Clock Drift & 0.4\% & - & Maximum oscillator deviation \\
        Bit Rate & 40 -- 640 & kbps & Adaptive link speed \\
        \bottomrule
    \end{tabular}
\end{table}

\end{document}
