%%%%%%%%%%%%%%%%%%%%%%%%%%%%%%%%%%%%%%%%%%%%%%%%%%%%%%%%%%%
% --------------------------------------------------------
% Main Paper Template
% --------------------------------------------------------
% 注意:请确保 main.cls, rhobabel.sty, rhoenvs.sty, ref.bib
% 在同一目录下。
%%%%%%%%%%%%%%%%%%%%%%%%%%%%%%%%%%%%%%%%%%%%%%%%%%%%%%%%%%%

% 加载主论文文档类 (双栏,9pt)
\documentclass[9pt,a4paper,twoside,twocolumn]{main}

\usepackage[english]{babel}
\usepackage{layout}

% --------------------------------------------------------
% 文章信息配置
% --------------------------------------------------------

% 期刊名或预印本信息 (首页顶部)
% \journalname{Preprint to Techrxiv} % 如果需要显示,取消注释

% 论文标题
\title{Paper Title: Input your title here}

% --------------------------------------------------------
% 作者信息 (支持多作者,多机构)
% --------------------------------------------------------

\author[1]{First Author \orcidlink{0000-0000-0000-0000}}
\author[1,2]{Second Author \orcidlink{0000-0000-0000-0000}}
\author[1]{Third Author}

% 机构信息
\affil[1]{School of Computer Science and Technology, University Name, City, Country}
\affil[2]{Key Laboratory of Research, City, Country}

% 顶部 Running Header 作者缩写
\leadauthor{First Author et al.}

% --------------------------------------------------------
% 底部信息栏 (通信作者,第一作者,DOI 等)
% --------------------------------------------------------

% 页脚机构信息
\institution{University Name}

% 底部信息
\corres{Second Author}           % 通信作者
\email{corresponding@example.com}      % 邮箱
\firstauthor{First Author}    % 第一作者
\femail{first.author@example.com}     % 第一作者邮箱 (可选)

% \doi{DOI: 10.1109/TEMP.2025.1234567} % DOI (可选)
% \received{March 20, 2024} % 接收日期 (可选)

% --------------------------------------------------------
% 开启功能模块
% --------------------------------------------------------

\setbool{rho-abstract}{true}  % 开启摘要显示
\setbool{corres-info}{true}   % 开启底部信息栏显示

% --------------------------------------------------------
% 摘要 (必须放在 \maketitle 之前)
% --------------------------------------------------------
\begin{abstract}
Missing tag identification is critical for large-scale RFID systems. Existing approaches often suffer from low identification efficiency and reliability in noisy environments. To address these challenges, this paper proposes a novel protocol. By leveraging bit-level synchronization and voting mechanisms, our method tolerates clock drifts and channel errors. Experimental results demonstrate that the proposed method significantly outperforms state-of-the-art protocols in terms of throughput and accuracy, achieving robust performance even under unstable channel conditions. The source code and datasets are available online to facilitate further research in this domain.
\end{abstract}

% 关键词
\keywords{RFID, Missing Tag Identification, De-slotted Architecture, Perfect Hashing, Link-adaptive.}

% --------------------------------------------------------
% 正文开始
% --------------------------------------------------------

\begin{document}
	
\maketitle
\thispagestyle{firststyle} % 首页使用独特样式 (无页眉横线)

% --------------------------------------------------------
% 目录 (可选)
% --------------------------------------------------------
% \tableofcontents
% \vspace{10pt}
% \noindent\hrulefill
% \vspace{10pt}

% --------------------------------------------------------
% 引言
% --------------------------------------------------------
\section{Introduction}

\rhostart{R}{adio} Frequency Identification (RFID) technology has been widely adopted in supply chain management. \cite{Industry4} The quick and reliable identification of missing tags is a fundamental requirement for automated inventory systems. However, traditional slot-based ALOHA protocols introduce significant overhead due to guard times and synchronization requirements. This paper introduces a novel de-slotted architecture that eliminates these inefficiencies.

% --------------------------------------------------------
% 示例:图片
% --------------------------------------------------------
\begin{figure}[!t]
    \centering
    % 使用 Template 文件夹中现有的示例图片
    \includegraphics[width=0.8\linewidth]{Figure_Sample_1.pdf}
    \caption{Performance comparison between proposed method and baseline protocols. The proposed method achieves higher recall rates under varying bit error rates.}
    \label{fig:sample}
\end{figure}

% --------------------------------------------------------
% 示例:多级标题
% --------------------------------------------------------
\subsection{System Model}
We consider a standard RFID system consisting of a reader and a set of passive tags. The reader communicates with tags using the EPC Gen2 standard.

\subsubsection{Channel Constraints}
The wireless channel is assumed to be imperfect, subject to fading and noise. We model the bit errors using a Bernoulli process.

% --------------------------------------------------------
% 示例:表格 (带隔行变色)
% --------------------------------------------------------
\begin{table}[!t]
    \centering
    \caption{Simulation Parameters.}
    \label{tab:sample}
    \rowcolors{2}{gray!10}{white} % 开启隔行变色: 从第2行开始,奇数行灰,偶数行白
    \begin{tabular}{ccc}
        \toprule
        \textbf{Parameter} & \textbf{Symbol} & \textbf{Value} \\
        \midrule
        Tag Population & $N$ & 100 -- 1000 \\
        Frame Length & $L$ & 256 bits \\
        Error Rate & $P_e$ & 0.01 -- 0.1 \\
        \bottomrule
    \end{tabular}
\end{table}

% --------------------------------------------------------
% 参考文献
% --------------------------------------------------------
\printbibliography

\end{document}
