%%%%%%%%%%%%%%%%%%%%%%%%%%%%%%%%%%%%%%%%%%%%%%%%%%%%%%%%%%%
% Rho Class - Supplementary Material Template
% --------------------------------------------------------
% 使用说明:
%   1. 使用 class/sup 文档类(单栏排版)
%   2. 在导言区设置标题、作者、日期、版本等
%   3. 用 \setupxr{Main} 交叉引用正文中的图表编号
%      (需先编译 Main.tex 生成 Main.aux)
%   4. 支持摘要 (\begin{abstract}...\end{abstract})
%   5. 支持目录 (\tableofcontents)
%
% 编译方式: xelatex Supplementary → biber Supplementary → xelatex x2
% --------------------------------------------------------
%%%%%%%%%%%%%%%%%%%%%%%%%%%%%%%%%%%%%%%%%%%%%%%%%%%%%%%%%%%

\documentclass[10pt,a4paper,onecolumn]{class/sup}

%----------------------------------------------------------
% 基本宏包配置
%----------------------------------------------------------
\usepackage[english]{babel}
\usepackage{graphicx}
\usepackage{booktabs}
\usepackage[table]{xcolor}

%----------------------------------------------------------
% 主题色配置 (可选: red, purple, blue)
%----------------------------------------------------------
\rhotheme{red}

%----------------------------------------------------------
% 标题信息
%----------------------------------------------------------
\title{Supplementary Material: Detailed Derivations and Additional Results}
\author{First Author, Second Author, Corresponding Author}
\dates{\today}
\paperversion{Version -- \today}          % 页脚显示版本号

%----------------------------------------------------------
% 交叉引用正文(可选)
% 使用 \setupxr{Main} 后,可用 \ref{正文中的label} 引用正文图表
% 前提:Main.tex 已编译生成 Main.aux
%----------------------------------------------------------
\setupxr{Main}

%----------------------------------------------------------
% 页脚信息
%----------------------------------------------------------
\institution{University Name}
\leadauthor{First Author et al.}

%----------------------------------------------------------
% 摘要(放在 \begin{document} 之前)
%----------------------------------------------------------
\setbool{rho-abstract}{true}

\begin{abstract}
    This supplementary material provides detailed derivations of the mathematical models presented in the main text, specifically the convergence proof in Section 3. Additionally, we present extended experimental results comparing the proposed method with two additional baselines under varying channel conditions.
\end{abstract}

%----------------------------------------------------------
% 正文开始
%----------------------------------------------------------
\begin{document}

\maketitle

%% 目录(可选,适合较长的补充材料)
\tableofcontents
\vspace{20pt}
\hrule
\vspace{20pt}

\section{Overview}
\rhostart{T}{his} supplementary material accompanies the main manuscript. It includes detailed mathematical derivations, additional experimental results, and extended discussions that were not included in the main text due to space limitations.

\textbf{Template Note:} This document uses the \texttt{class/sup} class, which is optimized for supplementary materials:
\begin{itemize}
    \item \textbf{Single Column}: Defaults to \texttt{onecolumn} for better readability of large tables and equations.
    \item \textbf{Simplified Header}: Removes the complex author/affiliation block in favor of a clean title.
    \item \textbf{Cross-reference}: Use \texttt{$\backslash$setupxr\{Main\}} to reference figures/tables from the main text.
    \item \textbf{Version Info}: Displayed automatically in the footer via \texttt{$\backslash$paperversion}.
\end{itemize}

\section{Additional Methodology Details}
This section provides a deepening of the methods described in the main text.

\subsection{Algorithm Analysis}
The complexity of the proposed algorithm is $O(n \log n)$, as derived from standard principles \cite{Textbook_Example}. Below is the detailed proof.
\begin{equation}
    E = mc^2 + \int_{0}^{\infty} e^{-x} dx
\end{equation}

\section{Extended Results}

\subsection{Additional Figures}
We provide additional comparisons in Figure \ref{fig:sup_fig}.

\begin{figure}[h]
    \centering
    \includegraphics[width=0.6\linewidth]{abstract_graph.pdf}
    \caption{Supplementary Figure: detailed breakdown of the efficient frontier. The data correlates with the findings in Section 4 of the main manuscript.}
    \label{fig:sup_fig}
\end{figure}

\subsection{Raw Data Tables}
Table \ref{tab:sup_table} lists the complete dataset used for the simulation.

\begin{table}[h]
    \centering
    \caption{Complete Simulation Parameters and Results}
    \label{tab:sup_table}
    \rowcolors{2}{gray!10}{white}
    \begin{tabular}{lcccc}
        \toprule
        \textbf{Dataset} & \textbf{Parameter A} & \textbf{Parameter B} & \textbf{Precision} & \textbf{Recall} \\
        \midrule
        Set 1 & 0.5 & 100 & 98.2\% & 97.5\% \\
        Set 2 & 0.6 & 200 & 97.8\% & 96.9\% \\
        Set 3 & 0.7 & 300 & 96.5\% & 96.1\% \\
        Set 4 & 0.8 & 400 & 95.1\% & 95.5\% \\
        \bottomrule
    \end{tabular}
\end{table}

\printbibliography

\end{document}
