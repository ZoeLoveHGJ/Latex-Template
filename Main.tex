%%%%%%%%%%%%%%%%%%%%%%%%%%%%%%%%%%%%%%%%%%%%%%%%%%%%%%%%%%%
% Rho Class - Academic Paper Template (Standardized)
% --------------------------------------------------------
% 使用说明:
% 1. 本模板基于 Rho Class,适用于学术论文写作
% 2. 默认双栏排版 (twocolumn)
% 3. 请确保使用 BibLaTeX + Biber 进行参考文献编译
% 4. 核心类文件位于 class/ 目录下
% --------------------------------------------------------
%%%%%%%%%%%%%%%%%%%%%%%%%%%%%%%%%%%%%%%%%%%%%%%%%%%%%%%%%%%

\documentclass[10pt,a4paper,twoside,onecolumn]{class/main}

%----------------------------------------------------------
% 基本宏包配置
%----------------------------------------------------------
\usepackage[english]{babel}
\usepackage{graphicx}
\usepackage{booktabs}
\usepackage[table]{xcolor}

%----------------------------------------------------------
% 主题色配置 (可选: red, purple, blue)
%----------------------------------------------------------
\rhotheme{red}  % 切换主题色:red / purple / blue

%----------------------------------------------------------
% 论文信息配置 (Metadata)
%----------------------------------------------------------

% 标题
\title{LODS-MTI: A Link-Adaptive, 0rthogonal, and De-slotted Protocol for Robust and Fast RFID Missing Tag Identification}
% \journalname{Preprint Submitted to Journal Name}
% \paperversion{Draft Version 1.0}

% 作者信息
% 作信息
\author[1]{Hongquan Zhou$^{\orcidlink{0000-0000-0000-0000}}$}
\author[1,2]{Zhong Du$^{\orcidlink{0000-0000-0000-0000}}$}
\author[2,$*$]{Xiaolin Jia$^{*,\orcidlink{0000-0000-0000-0000}}$}
\author[1]{Yajun Gu}
\author[2]{Hong Yang}

% 机构信息
\affil[1]{Department of Computer Science, University Name, City, Country}
\affil[2]{Research Institute, Technology Center, City, Country}

% 通讯作者及其他元数据
\corres{Xiaolin Jia}
\email{my\_jiaxl@163.com}
\firstauthor{Hongquan ZHou}
\femail{2024319433@qq.com}
% \allemails{first@example.com, second@example.com}

% \dates{Received: January 01, 2025; Revised: February 15, 2025; Accepted: March 01, 2025}
% \doi{\url{https://doi.org/10.1000/xyz123}}

% 致谢/资助信息 (Acknowledgments / Funding)
% 这部分信息显示在首页底部
% \license{This work was supported by the National Natural Science Foundation of China (Grant No. 12345678) and the Fundamental Research Funds for the Central Universities. The authors would like to thank Dr. X and Prof. Y for their valuable suggestions. This work is licensed under a Creative Commons Attribution 4.0 International License, which permits use, sharing, adaptation, distribution and reproduction in any medium or format, as long as you give appropriate credit to the original author(s) and the source.}

% 页脚信息
\leadauthor{Hongquan ZHou et al.}
\institution{Southwest University of Science and Technology}

% 开启功能模块
\setbool{rho-abstract}{true}
\setbool{corres-info}{true}

%----------------------------------------------------------
% 摘要 (Abstract)
%----------------------------------------------------------
\begin{abstract}
Existing Missing Tag Identification (MTI) protocols trade logic-layer scheduling reliability for physical-layer spectral efficiency. This paper presents LODS-MTI, a Link-adaptive Orthogonal De-slotted protocol that eliminates inter-tag guard times using batch-synchronized continuous bitstreams. To mitigate synchronization loss in de-slotted transmissions, a digital majority voting mechanism corrects single-bit timing slips and accommodates 0.4\% clock drift. The design uses short-term coherence through ``Power-of-2'' batching to replace arithmetic division with bitwise operations. A tolerance-driven feedback loop ($\epsilon=0.30$) adjusts redundancy based on link quality to maintain a Safe Operating Area (SOA) with 99.10\% accuracy within 128-bit windows. Experimental results show that LODS-MTI achieves a normalized time efficiency of $\Gamma \approx 3.52$ (a 50\% increase over baselines) and sustains 5,200--8,811 tags/s goodput ($P_{e}\approx0.1$) at $150~\mu J/tag$, reducing energy consumption by approximately 10$\times$ compared to analog collision resolution schemes. Code and results are available at \href{https://github.com/ZoeLoveHGJ/Project_LODS_MTI}{GitHub}.
\end{abstract}
\keywords{RFID, Missing Tag Identification, De-slotted Architecture, Perfect Hashing, Link-adaptive.}

%----------------------------------------------------------
% 图形化摘要 (Graphical Abstract)
%----------------------------------------------------------
\graphicalabstract{\includegraphics[width=0.8\linewidth]{abstract_graph.pdf}}{This graphical abstract illustrates the architecture and performance of LODS-MTI, a protocol designed to reconcile logical reliability with physical spectral efficiency in RFID systems. 1. Context: Resolves the structural conflict between stable logic-layer scheduling (Type I) and fast physical-layer collision resolution (Type II). 2. Mechanism: Replaces guard intervals with a de-slotted continuous stream for spectral compression. A Digital Majority Voting logic ($\sum w_i \ge \theta$) recovers data from synchronization drift, achieving $\Gamma \approx 3.52$. 3. Outcome: Delivers 10x lower energy ($\sim 150 \mu J/tag$) and sustains 5,200+ tags/s goodput with 99.1\% accuracy.}

%----------------------------------------------------------
% 正文开始
%----------------------------------------------------------
\begin{document}

\maketitle
\clearpage % 强制Introduction从第二页开始
% \rhostart{W}{elcome}            
\section{Introduction}
Welcome to the \textbf{Rho Class Standardized Template}. This template is designed for academic preprints and articles, featuring a clean, professional "Academic Navy" theme. It is built on the `extarticle` class and fully supports XeLaTeX compilation.

\textbf{Quick Start Guide:}
\begin{enumerate}
    \item \textbf{Configuration}: Open `Main.tex` and edit the metadata block (Title, Authors, Affiliations) at the top.
    \item \textbf{Class Files}: The core logic resides in the `class/` directory (`main.cls`, `sup.cls`, `cover.cls`). Do not modify these unless you need to change the global style.
    \item \textbf{Compilation}: Use \textbf{XeLaTeX} + \textbf{Biber} (for bibliography) + \textbf{XeLaTeX} sequence.
\end{enumerate}

\section{Template Features}

\subsection{First Page Elements}
\begin{itemize}
    \item \textbf{Header}: Displays the Running Title (`\title`) and Lead Author (`\leadauthor`).
    \item \textbf{Footer}: Contains the Institution Name (`\institution`), Preprint Version Date (automatically generated), and Page Number (e.g., "1-5").
    \item \textbf{Metadata}: The bottom rule contains the Correspondence (`\corres`), License/Funding (`\license`), and Dates (`\dates`). Ensure `\setbool{corres-info}{true}` is set to enable this block.
\end{itemize}

\subsection{Graphical Abstract}
You can include a graphical abstract before the introduction using the `\graphicalabstract` command.
\begin{verbatim}
\graphicalabstract{
    \includegraphics[width=0.8\linewidth]{abstract.pdf}
    \captionof{figure}{...}
}
\end{verbatim}
\subsection{Citations and References}
This template uses `biblatex` with `biber` backend.
\begin{itemize}
    \item Add your references to `ref.bib`.
    \item Cite them using `\cite{key}`, e.g., \cite{Industry4} or \cite{Techrxiv}.
\end{itemize}

\subsection{Figures and Tables}
Figures should be placed in the `figures/` directory. Use standard LaTeX environments.
\begin{figure}[h]
    \centering
    \includegraphics[width=0.8\linewidth]{abstract_graph.pdf}
    \caption{Example of a figure. Note that the caption is left-aligned and colored in Academic Navy.}
    \label{fig:example}
\end{figure}

Table \ref{tab:example} demonstrates the default style.

\begin{table}[h]
    \centering
    \caption{Example Table Style}
    \label{tab:example}
    \rowcolors{2}{gray!10}{white}
    \begin{tabular}{ccc}
        \toprule
        \textbf{Feature} & \textbf{Command} & \textbf{Note} \\
        \midrule
        Lead Author & \texttt{\textbackslash leadauthor} & Header usage \\
        Institution & \texttt{\textbackslash institution} & Footer usage \\
        License & \texttt{\textbackslash license} & Bottom info \\
        \bottomrule
    \end{tabular}
\end{table}

\section{Conclusion}
This template aims to provide an "out-of-the-box" experience for professional academic writing. If you encounter missing footer information, please check that `\setbool{corres-info}{true}` is present in your preamble.

%----------------------------------------------------------
% 参考文献
%----------------------------------------------------------
\printbibliography

\end{document}
